\documentclass{article}
\usepackage[utf8]{inputenc}

\title{Agile Software Projects}
\author{}
\date{November 2020}

\begin{document}

\maketitle

\section{Aims and Objectives}
\subsection{Product Goals \& Value Proposition}
\emph{Set clear goals and concise and appropriate challenges which are measurable. Aims should be specific, with your objectives building up a bigger picture of what you hope to do. Goals and operations should be clearly specified.}

\subsection{Usability Goals}
\textit{The theme of usability is key.}

\subsection{Measuring Success}
\textit{This section should set out the measuring criteria for your work and your presentation will be judged in relation to these aims and objectives.}


\section{Planning and Requirements Gathering}
\subsection{Methodology}
\textit{You should focus on how you hope to plan and gather requirements. We discussed some different methods in the lectures and discussion group activities. It is important that you evaluate the different methods in making your decisions and provide some written analysis in your reports, with clear critique and some functional understanding for the higher marks.}
\subsection{High-Level Requirements Funnel}

\subsection{Implementation Approach}

\section{Formative Testing and Evaluation}
\subsection{Identifying Users}
\textit{This is the part of the project which requires the most research and understanding. For middle marks, you should provide a good, clear overview of techniques to identify and sample users, with a set of contextually relevant information and a balanced overall argument. This should bridge cohesively with your requirements engineering techniques, as well as your general research in target demographics, types of systems and processes involved}
\subsection{Testing \& Evaluation Methods}
\textit{Higher marks would require rigorous tests that are insightful and utilise a wide range of metrics to analyse success through different lenses.}


\subsection{}

\section{Prototyping Techniques}
\textit{You should describe your prototype, where strengths in different techniques lie and where they are used. This section should also detail the movement between low- fidelity and high-fidelity techniques, with a view to building the system and the types of technology involved. There should be evidence of iterative design and evaluation steps.}
\subsection{Techniques}
\subsection{Low-Fidelity Prototypes}
\subsection{High-Fidelity Prototypes}

\section{Evaluation Techniques}
\subsection{Critical Success Factors}
\textit{Where do the critical success factors lie and what novel techniques could you use to measure your work?}

\subsection{Measurement of Success \& Failure}
\textit{How do you intend to measure success and failure in context? }

\subsection{Results}
\textit{ What works well about the system and where are the fundamental flaws?}


\end{document}
